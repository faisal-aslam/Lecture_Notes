\documentclass[12pt]{article}
\usepackage{amsmath, amssymb}
\usepackage{physics}
\usepackage{hyperref}

\title{Measurement in Non-Standard Bases \\ Change of Basis and Quantum Measurements}
\author{Quantum Computing Course}
\date{}

\begin{document}

\maketitle

\section{Motivation}

In previous lectures, we studied quantum measurement in the \emph{computational basis}
\[
\{ \ket{0}, \ket{1} \}.
\]
However, quantum mechanics allows measurement in \emph{any orthonormal basis}.
In this lecture, we extend the notion of measurement to non-standard bases and show how such measurements are implemented using change of basis.

Throughout this lecture, we restrict ourselves to:
\begin{itemize}
    \item Projective measurements
    \item Finite-dimensional Hilbert spaces
    \item Standard basis measurement as the physical primitive
\end{itemize}

\bigskip

\section{Measurement in an Arbitrary Orthonormal Basis}

Let $\{ \ket{\phi_0}, \ket{\phi_1} \}$ be an orthonormal basis of a single-qubit Hilbert space.

A measurement in this basis is described by the projectors:
\[
P_0 = \ket{\phi_0}\bra{\phi_0}, \qquad
P_1 = \ket{\phi_1}\bra{\phi_1}.
\]

For a quantum state $\ket{\psi}$, the probability of outcome $i$ is
\[
p(i) = \bra{\psi} P_i \ket{\psi} = |\braket{\phi_i | \psi}|^2,
\]
and the post-measurement state is
\[
\ket{\psi_i} = \frac{P_i \ket{\psi}}{\sqrt{p(i)}}.
\]

\bigskip

\section{The $\ket{+}, \ket{-}$ Basis}

The most important non-standard basis for a qubit is the \emph{Hadamard basis}:
\[
\ket{+} = \frac{1}{\sqrt{2}}(\ket{0} + \ket{1}), \qquad
\ket{-} = \frac{1}{\sqrt{2}}(\ket{0} - \ket{1}).
\]

These states satisfy:
\[
\braket{+|+} = 1, \quad
\braket{-|-} = 1, \quad
\braket{+|-} = 0.
\]

\bigskip

\section{Measurement in the $\ket{+}, \ket{-}$ Basis}

Let
\[
\ket{\psi} = \alpha \ket{0} + \beta \ket{1},
\quad |\alpha|^2 + |\beta|^2 = 1.
\]

Then:
\[
\braket{+|\psi} = \frac{\alpha + \beta}{\sqrt{2}}, \qquad
\braket{-|\psi} = \frac{\alpha - \beta}{\sqrt{2}}.
\]

Thus, the measurement probabilities are:
\[
p(+) = \frac{|\alpha + \beta|^2}{2}, \qquad
p(-) = \frac{|\alpha - \beta|^2}{2}.
\]

\bigskip

\section{Change of Basis as a Unitary Transformation}

Let $U$ be a unitary operator such that:
\[
U \ket{0} = \ket{\phi_0}, \qquad
U \ket{1} = \ket{\phi_1}.
\]

Then measurement in the basis $\{ \ket{\phi_0}, \ket{\phi_1} \}$ can be implemented as:
\begin{enumerate}
    \item Apply $U^\dagger$
    \item Measure in the computational basis
\end{enumerate}

\subsection*{Justification}

\[
\braket{\phi_i | \psi}
=
\braket{i | U^\dagger | \psi},
\]
so the measurement probabilities are preserved.

\bigskip

\section{Hadamard Gate as a Basis-Change Operator}

The Hadamard gate is defined as:
\[
H = \frac{1}{\sqrt{2}}
\begin{pmatrix}
1 & 1 \\
1 & -1
\end{pmatrix}.
\]

It satisfies:
\[
H \ket{0} = \ket{+}, \qquad
H \ket{1} = \ket{-},
\]
and
\[
H^\dagger = H.
\]

Thus, measuring in the $\ket{+}, \ket{-}$ basis is equivalent to:
\begin{enumerate}
    \item Apply $H$
    \item Measure in the computational basis
\end{enumerate}

\bigskip

\section{Circuit Representation of Basis-Change Measurement}

\[
\ket{\psi}
\xrightarrow{\;H\;}
H\ket{\psi}
\xrightarrow{\;\text{measurement}\;}
\text{classical outcome}
\]

\begin{quote}
\emph{Quantum hardware measures only in the computational basis; all other measurements are implemented via unitary change of basis.}
\end{quote}

\bigskip

\section{Partial Measurement in a Non-Standard Basis}

Consider a two-qubit state $\ket{\Psi}_{AB}$.

To measure qubit $A$ in the $\ket{+}, \ket{-}$ basis:
\begin{enumerate}
    \item Apply $H$ to qubit $A$
    \item Measure qubit $A$ in the computational basis
\end{enumerate}

The post-measurement state of qubit $B$ depends on the measurement outcome.

\bigskip

\section{Comparison: Standard vs Non-Standard Basis Measurement}

\begin{center}
\begin{tabular}{|c|c|}
\hline
Standard Basis & Non-Standard Basis \\
\hline
Measure $\ket{0}, \ket{1}$ & Measure $\ket{\phi_0}, \ket{\phi_1}$ \\
No preprocessing & Apply unitary $U^\dagger$ first \\
Direct probabilities & Probabilities via overlaps \\
\hline
\end{tabular}
\end{center}

\bigskip

\section{Key Takeaways}

\begin{itemize}
    \item Measurement can be performed in any orthonormal basis
    \item Measurement probabilities depend on state overlaps
    \item All non-standard measurements reduce to standard-basis measurement
    \item Change of basis is implemented using unitary operators
    \item The Hadamard gate enables $\ket{+}, \ket{-}$ basis measurement
\end{itemize}

\bigskip

\section{Looking Ahead}

Non-standard basis measurements are essential for:
\begin{itemize}
    \item Quantum algorithms (Deutsch--Jozsa, Simon, Grover)
    \item Quantum teleportation
    \item Bell inequality tests
    \item Quantum error correction
\end{itemize}

\end{document}
