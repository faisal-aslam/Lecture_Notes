\documentclass[12pt]{article}
\usepackage{amsmath, amssymb}
\usepackage{graphicx}
\usepackage{tcolorbox}
\usepackage{array}
\usepackage{booktabs}
\usepackage{hyperref}
\usepackage{geometry}
\geometry{margin=1in}

\title{Measurement in Non-Standard Bases \\ Change of Basis and Quantum Measurements}
\author{Quantum Computing Course}
\date{}

% Manual definitions to replace physics package issues
\newcommand{\ket}[1]{\left|#1\right\rangle}
\newcommand{\bra}[1]{\left\langle#1\right|}
\newcommand{\braket}[2]{\left\langle #1 \middle| #2 \right\rangle}
\newcommand{\braopket}[3]{\left\langle #1 \left| #2 \right| #3 \right\rangle}

\begin{document}

\maketitle

\tableofcontents

\newpage

\section{Mathematical Prerequisites: Hilbert Space}

\subsection{What is Hilbert Space? (Layman's Explanation)}

\begin{tcolorbox}[title=Hilbert Space Simplified]
A \textbf{Hilbert space} is the mathematical arena where quantum mechanics happens. Think of it as:
\begin{itemize}
    \item A \textbf{generalized vector space} (like 3D space, but potentially infinite-dimensional)
    \item With \textbf{inner products} that give us "angles" and "lengths" between states
    \item Complete (no "holes" - every convergent sequence has a limit)
\end{itemize}
\end{tcolorbox}

\textbf{Why Hilbert space for quantum mechanics?}
\begin{itemize}
    \item Quantum states are \textbf{vectors} in Hilbert space
    \item Superpositions are just \textbf{vector sums} in this space
    \item Measurements correspond to \textbf{projections} onto basis vectors
    \item Time evolution is described by \textbf{rotations} (unitary operations) in this space
\end{itemize}

\textbf{Single-qubit Hilbert space:}
- 2-dimensional complex vector space
- Basis: $\{\ket{0}, \ket{1}\}$
- Any state: $\alpha\ket{0} + \beta\ket{1}$ with $|\alpha|^2 + |\beta|^2 = 1$

\subsection{Observables: What We Actually Measure}

\begin{tcolorbox}[title=Observable Definition]
An \textbf{observable} in quantum mechanics is any physical quantity that can be measured. Mathematically:
\begin{itemize}
    \item Represented by a \textbf{Hermitian operator} $A$ (equal to its conjugate transpose: $A = A^\dagger$)
    \item Has \textbf{real eigenvalues} (the possible measurement outcomes)
    \item Has \textbf{orthogonal eigenvectors} (the states that give definite outcomes)
\end{itemize}
\end{tcolorbox}

\textbf{Key examples:}
- Pauli $X$ operator: $X = \begin{pmatrix}0&1\\1&0\end{pmatrix}$ with eigenvalues $\pm1$, eigenvectors $\ket{+}, \ket{-}$
- Pauli $Z$ operator: $Z = \begin{pmatrix}1&0\\0&-1\end{pmatrix}$ with eigenvalues $\pm1$, eigenvectors $\ket{0}, \ket{1}$
- Energy operator (Hamiltonian): $H$ with energy levels as eigenvalues

\textbf{Why Hermitian?}
\begin{enumerate}
    \item Real eigenvalues $\Rightarrow$ measurement results are real numbers
    \item Orthogonal eigenvectors $\Rightarrow$ distinguishable measurement outcomes
    \item Completeness $\Rightarrow$ any state can be expressed in the eigenbasis
\end{enumerate}

\section{Motivation: Beyond the Computational Basis}

\subsection{Limitations of Computational Basis Measurements}

In previous lectures, we studied quantum measurement exclusively in the \textbf{computational basis}:
\[
\{ \ket{0}, \ket{1} \}.
\]
While this basis is natural for digital computation and hardware implementation, it represents only one possible way to extract information from a quantum system.

\begin{tcolorbox}[title=Key Question]
If quantum states can exist in superpositions like $\alpha\ket{0} + \beta\ket{1}$, why should we only measure whether they're "0" or "1"? What about measuring whether they're "+" or "-"?
\end{tcolorbox}

Quantum mechanics permits measurement in \textbf{any orthonormal basis}. This freedom is not just mathematical---it's essential for:
\begin{itemize}
    \item Quantum algorithms that exploit interference patterns
    \item Testing quantum correlations (Bell inequalities)
    \item Quantum error correction syndromes
    \item Quantum state tomography
\end{itemize}

\subsection{Physical Implementation Reality}

All physical quantum hardware has a \textbf{native measurement basis}, typically the energy eigenbasis (which corresponds to $\{\ket{0}, \ket{1}\}$ for qubits). This is because:
\begin{itemize}
    \item Measurement apparatus couples to specific physical observables (e.g., energy, charge, flux)
    \item The computational basis is often the easiest to distinguish physically
    \item Decoherence typically occurs toward this basis
\end{itemize}

\begin{tcolorbox}[title=Hardware Constraint]
\textbf{Quantum computers measure only in their native computational basis.} All other measurements must be implemented via change of basis operations before measurement.
\end{tcolorbox}

\subsection{Applications Requiring Non-Standard Measurements}

\begin{itemize}
    \item \textbf{Deutsch-Jozsa algorithm:} Requires final measurement in $\{\ket{+}, \ket{-}\}$ basis
    \item \textbf{Bell inequality tests:} Measurement in rotated bases reveals quantum correlations
    \item \textbf{Quantum teleportation:} Bell basis measurements are crucial
    \item \textbf{Quantum key distribution:} Security relies on measurements in complementary bases
\end{itemize}

\section{The Born Rule: Why it Works}

\subsection{The Fundamental Postulate of Quantum Measurement}

\begin{tcolorbox}[title=The Born Rule (Modern Statement)]
For a quantum system in state $\ket{\psi}$ and an observable $A$ with eigenvectors $\{\ket{\phi_i}\}$ (eigenvalues $\lambda_i$), the probability of obtaining measurement outcome $\lambda_i$ is:
\[
p(\lambda_i) = \left| \braket{\phi_i}{\psi} \right|^2
\]
\end{tcolorbox}

\subsection{Why $|\langle \phi_i | \psi \rangle|^2$? The Logic Behind}

\begin{enumerate}
    \item \textbf{Projection Interpretation:}
    $\braket{\phi_i}{\psi}$ is the "overlap" or "projection" of $\ket{\psi}$ onto $\ket{\phi_i}$.
    If $\ket{\psi} = \ket{\phi_i}$ (perfectly aligned), then $\braket{\phi_i}{\psi} = 1$, so probability = 1.

    \item \textbf{Consistency Requirement:}
    Probabilities must be non-negative real numbers between 0 and 1.
    - $\braket{\phi_i}{\psi}$ is complex in general
    - $|\braket{\phi_i}{\psi}|^2$ is always real and non-negative

    \item \textbf{Probability Conservation:}
    Since $\sum_i |\braket{\phi_i}{\psi}|^2 = \braket{\psi}{\psi} = 1$, total probability = 1.

    \item \textbf{Connection to Classical Probability:}
    In classical probability: $p(i) = \frac{\text{"size" of event }i}{\text{total "size"}}$
    In quantum: The "size" is replaced by squared projection magnitudes.
\end{enumerate}

\subsection{Historical Context and Experimental Verification}

Max Born (1926) proposed this rule to interpret Schrödinger's wavefunction. It was controversial initially but has been verified in countless experiments:

\begin{itemize}
    \item \textbf{Double-slit experiment:} Probability of particle hitting screen = $|\psi(x)|^2$
    \item \textbf{Stern-Gerlach experiment:} Spin measurement probabilities follow Born rule
    \item \textbf{Quantum optics:} Photon detection statistics confirm Born rule
\end{itemize}

\begin{tcolorbox}[title=Deep Insight]
The Born rule bridges the deterministic Schrödinger equation (continuous evolution) with probabilistic measurement outcomes (discrete events). It's the \textbf{collapse postulate} that converts quantum possibilities into classical actualities.
\end{tcolorbox}

\section{Mathematical Formalism of Measurement in Arbitrary Bases}

\subsection{Projective Measurements in Orthonormal Bases}

Let $\{ \ket{\phi_0}, \ket{\phi_1} \}$ be an orthonormal basis for a single-qubit Hilbert space. Orthonormality and completeness are expressed as
\[
\braket{\phi_i}{\phi_j} = \delta_{ij},
\qquad
\sum_{i=0}^1 \ket{\phi_i}\bra{\phi_i} = I.
\]

A \textbf{projective measurement} in this basis is described by the projection operators
\[
P_0 = \ket{\phi_0}\bra{\phi_0},
\qquad
P_1 = \ket{\phi_1}\bra{\phi_1}.
\]

These projectors satisfy
\[
P_i^2 = P_i, \qquad P_i P_j = 0 \ (i \neq j), \qquad P_0 + P_1 = I.
\]

\subsection{The Born Rule in an Arbitrary Basis}

Let the quantum state of the qubit be $\ket{\psi}$, with $\braket{\psi}{\psi} = 1$.
The probability of obtaining outcome $i$, corresponding to the basis state $\ket{\phi_i}$, is given by the Born rule:
\[
p(i) = \braopket{\psi}{P_i}{\psi}
     = \left| \braket{\phi_i}{\psi} \right|^2.
\]

This expression holds for measurements in \emph{any} orthonormal basis.

\subsection{Post-Measurement State Update}

If outcome $i$ is obtained, the post-measurement state is
\[
\ket{\psi_i} = \frac{P_i \ket{\psi}}{\sqrt{p(i)}} = \ket{\phi_i},
\]
up to an unobservable global phase.

Thus, a projective measurement in the basis
$\{ \ket{\phi_0}, \ket{\phi_1} \}$ prepares the system in the corresponding basis state, independent of the initial state.

\begin{tcolorbox}[title=Important Distinction]
\textbf{Computational-basis measurement:} Projects onto $\ket{0}$ or $\ket{1}$. \\[4pt]
\textbf{Arbitrary-basis measurement:} Projects onto $\ket{\phi_0}$ or $\ket{\phi_1}$.
\end{tcolorbox}

\section{The Hadamard Basis: $\{\ket{+}, \ket{-}\}$}

\subsection{Definition and Orthonormality}

The most important non-standard basis for qubits is the \textbf{Hadamard basis} (also called $X$-basis or diagonal basis):
\[
\ket{+} = \frac{1}{\sqrt{2}}(\ket{0} + \ket{1}), \qquad
\ket{-} = \frac{1}{\sqrt{2}}(\ket{0} - \ket{1}).
\]

These states form an orthonormal basis:
\[
\braket{+}{+} = 1, \quad
\braket{-}{-} = 1, \quad
\braket{+}{-} = 0.
\]

\subsection{Geometric Interpretation on the Bloch Sphere}

On the Bloch sphere representation:
\begin{itemize}
    \item $\ket{0}$ points to the \textbf{north pole} (+Z direction)
    \item $\ket{1}$ points to the \textbf{south pole} (-Z direction)
    \item $\ket{+}$ points to the \textbf{positive X-axis} (equator, $\phi=0$)
    \item $\ket{-}$ points to the \textbf{negative X-axis} (equator, $\phi=\pi$)
\end{itemize}

Thus, measuring in $\{\ket{+}, \ket{-}\}$ corresponds to measuring the \textbf{Pauli $X$ observable}:
\[
X = \ket{+}\bra{+} - \ket{-}\bra{-}.
\]

\subsection{Measurement Probabilities in the Hadamard Basis}

For a general state $\ket{\psi} = \alpha\ket{0} + \beta\ket{1}$:
\[
\braket{+}{\psi} = \frac{\alpha + \beta}{\sqrt{2}}, \qquad
\braket{-}{\psi} = \frac{\alpha - \beta}{\sqrt{2}}.
\]

The measurement probabilities are:
\[
p(+) = \frac{|\alpha + \beta|^2}{2}, \qquad
p(-) = \frac{|\alpha - \beta|^2}{2}.
\]

\subsubsection*{Example 1: Equal Superposition}
For $\ket{\psi} = \frac{1}{\sqrt{2}}(\ket{0} + \ket{1}) = \ket{+}$:
\[
p(+) = 1, \quad p(-) = 0.
\]
Measuring $\ket{+}$ in $\{\ket{+}, \ket{-}\}$ basis always yields "+".

\subsubsection*{Example 2: Computational Basis State}
For $\ket{\psi} = \ket{0}$:
\[
p(+) = \frac{|1 + 0|^2}{2} = \frac{1}{2}, \quad
p(-) = \frac{|1 - 0|^2}{2} = \frac{1}{2}.
\]
Equal probabilities---this makes sense since $\ket{0}$ is halfway between $\ket{+}$ and $\ket{-}$ on the Bloch sphere.

\section{Change of Basis as a Unitary Operation}

\subsection{The Basis-Change Theorem}

Let $\{\ket{\phi_0}, \ket{\phi_1}\}$ be an orthonormal basis. There exists a unitary operator $U$ such that:
\[
U\ket{0} = \ket{\phi_0}, \qquad U\ket{1} = \ket{\phi_1}.
\]
Equivalently, $U$ maps the computational basis to the new basis.

\begin{tcolorbox}[title=Construction of $U$]
Given $\ket{\phi_0} = a\ket{0} + b\ket{1}$ and $\ket{\phi_1} = c\ket{0} + d\ket{1}$ with $\braket{\phi_0}{\phi_1}=0$:
\[
U = \begin{pmatrix} a & c \\ b & d \end{pmatrix}
\]
More abstractly: $U = \ket{\phi_0}\bra{0} + \ket{\phi_1}\bra{1}$.
\end{tcolorbox}

\subsection{Equivalence of Measurements: Formal Proof}

\textbf{Theorem:} Measurement in basis $\{\ket{\phi_0}, \ket{\phi_1}\}$ is equivalent to:
\begin{enumerate}
    \item Apply $U^\dagger$ to the state
    \item Measure in computational basis $\{\ket{0}, \ket{1}\}$
\end{enumerate}

\textbf{Proof:}
The probability of outcome $\ket{\phi_i}$ in the original measurement is:
\[
p(i) = |\braket{\phi_i}{\psi}|^2.
\]
After applying $U^\dagger$, the state becomes $U^\dagger\ket{\psi}$. The probability of measuring $\ket{i}$ in computational basis is:
\[
|\braket{i}{U^\dagger}{\psi}|^2 = |\braket{\phi_i}{\psi}|^2 = p(i).
\]
Thus, the probabilities are identical. The post-measurement states are also equivalent (up to the basis change).

\subsection{General Recipe for Arbitrary Basis Measurement}

To measure state $\ket{\psi}$ in basis $\{\ket{\phi_0}, \ket{\phi_1}\}$:
\begin{enumerate}
    \item \textbf{Construct} $U$ such that $U\ket{0} = \ket{\phi_0}$, $U\ket{1} = \ket{\phi_1}$
    \item \textbf{Apply} $U^\dagger$ to $\ket{\psi}$
    \item \textbf{Measure} in computational basis $\{\ket{0}, \ket{1}\}$
    \item \textbf{Interpret:} Outcome 0 corresponds to $\ket{\phi_0}$, outcome 1 to $\ket{\phi_1}$
\end{enumerate}

\section{The Hadamard Gate as a Basis-Change Operator}

\subsection{Properties of $H$: Self-Inverse and Real}

The Hadamard gate:
\[
H = \frac{1}{\sqrt{2}}\begin{pmatrix} 1 & 1 \\ 1 & -1 \end{pmatrix}
\]
has key properties:
\begin{itemize}
    \item \textbf{Self-inverse:} $H^\dagger = H$, $H^2 = I$
    \item \textbf{Basis change:} $H\ket{0} = \ket{+}$, $H\ket{1} = \ket{-}$
    \item \textbf{Real and symmetric:} $H = H^\top = H^\dagger$
\end{itemize}

\subsection{Circuit Implementation for $\{\ket{+}, \ket{-}\}$ Measurement}

To measure in $\{\ket{+}, \ket{-}\}$ basis:
\[
\boxed{\ket{\psi} \longrightarrow H \longrightarrow \bigcirc\!\!\!\!\!M \longrightarrow \text{classical bit}}
\]

\begin{itemize}
    \item \textbf{Hardware executes:} Apply $H$, then measure in $\{\ket{0}, \ket{1}\}$
    \item \textbf{Interpretation:} Outcome 0 means $\ket{+}$, outcome 1 means $\ket{-}$
\end{itemize}

\subsection{Interpretation: $X$-basis Measurement via $HZH$}

An alternative perspective: Measuring $X$ (Pauli operator) on state $\ket{\psi}$:
\[
\braopket{\psi}{X}{\psi} = \braopket{\psi}{H Z H}{\psi}.
\]
This shows that measuring the expectation value of $X$ is equivalent to measuring $Z$ on $H\ket{\psi}$.

\section{General Basis Change: Constructing $U$ for Arbitrary Bases}

\subsection{Single-Qubit Case: Relation to Rotation Operators}

Any single-qubit basis $\{\ket{\phi_0}, \ket{\phi_1}\}$ can be obtained from $\{\ket{0}, \ket{1}\}$ by a rotation on the Bloch sphere. The unitary $U$ can be expressed as:
\[
U = e^{-i\theta/2} R_{\hat{n}}(\phi)
\]
for some axis $\hat{n}$ and angles $\theta, \phi$.

\subsubsection*{Example: Y-basis}
The $Y$-basis is $\{\ket{+i}, \ket{-i}\}$ where:
\[
\ket{+i} = \frac{1}{\sqrt{2}}(\ket{0} + i\ket{1}), \quad
\ket{-i} = \frac{1}{\sqrt{2}}(\ket{0} - i\ket{1}).
\]
The basis change unitary is $U = HS^\dagger$ where $S = \begin{pmatrix}1&0\\0&i\end{pmatrix}$.

\subsection{Multi-Qubit Bases and Tensor Product Structures}

For multi-qubit systems, we can measure each qubit in a different basis. The overall unitary is a tensor product:
\[
U_{\text{total}} = U_1 \otimes U_2 \otimes \cdots \otimes U_n.
\]

\subsubsection*{Example: Bell Basis Measurement}
The Bell basis for two qubits is:
\begin{align*}
\ket{\Phi^+} &= \frac{1}{\sqrt{2}}(\ket{00} + \ket{11}) \\
\ket{\Phi^-} &= \frac{1}{\sqrt{2}}(\ket{00} - \ket{11}) \\
\ket{\Psi^+} &= \frac{1}{\sqrt{2}}(\ket{01} + \ket{10}) \\
\ket{\Psi^-} &= \frac{1}{\sqrt{2}}(\ket{01} - \ket{10})
\end{align*}
Measurement in this basis requires $H$ on first qubit followed by CNOT.

\subsection{Non-Standard Bases in Higher Dimensions}

For a $d$-dimensional system, measurement in basis $\{\ket{\phi_i}\}_{i=0}^{d-1}$ uses unitary $U$ with columns $\ket{\phi_i}$:
\[
U = \sum_{i=0}^{d-1} \ket{\phi_i}\bra{i}.
\]

\section{Partial Measurement in Non-Standard Bases}

\subsection{Procedure for Single-Qubit Basis Change in Multi-Qubit Systems}

Consider a two-qubit state $\ket{\Psi}_{AB}$. To measure qubit $A$ in basis $\{\ket{\phi_0}, \ket{\phi_1}\}$:
\begin{enumerate}
    \item Apply $U_A^\dagger$ to qubit $A$ only (where $U_A\ket{0} = \ket{\phi_0}$, $U_A\ket{1} = \ket{\phi_1}$)
    \item Measure qubit $A$ in computational basis
    \item The post-measurement state of qubit $B$ depends on the outcome
\end{enumerate}

\subsection{Conditional States and Correlations}

\subsubsection*{Example: Bell State with $X$-basis Measurement}
Start with Bell state: $\ket{\Phi^+} = \frac{1}{\sqrt{2}}(\ket{00} + \ket{11})$.

Measure qubit 1 in $\{\ket{+}, \ket{-}\}$ basis:
\begin{enumerate}
    \item Apply $H$ to qubit 1: $H_1\ket{\Phi^+} = \frac{1}{2}(\ket{0+} + \ket{1-})$
    \item Measure qubit 1 in computational basis:
    \begin{itemize}
        \item Outcome 0: $\ket{0}_1$, remaining state: $\ket{+}_2$
        \item Outcome 1: $\ket{1}_1$, remaining state: $\ket{-}_2$
    \end{itemize}
\end{enumerate}
Thus, measuring one qubit of a Bell state in $X$-basis leaves the other in an $X$-basis eigenstate.

\subsection{Circuit Representation with Controlled Basis Changes}

For adaptive measurements where basis choice depends on previous outcomes:

\[
\boxed{\ket{\psi}_A \longrightarrow \bigcirc\!\!\!\!\!M \longrightarrow \text{classical control} \longrightarrow \boxed{U(\text{outcome})}_B \longrightarrow \bigcirc\!\!\!\!\!M}
\]

\section{Practical Considerations and Common Pitfalls}

\subsection{Hardware Limitations: Only Computational Basis is Physical}

\begin{itemize}
    \item \textbf{Gate errors:} $U^\dagger$ is imperfect (finite gate fidelity)
    \item \textbf{Timing:} Basis change adds circuit depth and decoherence time
    \item \textbf{Calibration:} Different bases may have different measurement fidelities
\end{itemize}

\subsection{Error Propagation through Basis Changes}

If $U$ has error $\epsilon$, the measurement probabilities acquire error:
\[
p_{\text{actual}}(i) = \left|\braket{i}{(U^\dagger + \delta U)}{\psi}\right|^2 \approx p_{\text{ideal}}(i) + 2\Re[\braopket{\psi}{U}{i}\braket{i}{\delta U}{\psi}].
\]

\subsection{Classical Post-Processing Interpretation}

An alternative viewpoint: Instead of physically changing basis, measure in computational basis and classically compute:
\[
p(i) = \left|\braket{\phi_i}{\psi}\right|^2 = \left|\sum_j U_{ij}^* \braket{j}{\psi}\right|^2.
\]
This requires knowing the full quantum state, which defeats the purpose of quantum computation.

\begin{tcolorbox}[title=Key Insight]
The power of quantum measurement comes from \textbf{physically} changing basis before measurement, not classical post-processing.
\end{tcolorbox}

\section{Comparison and Summary}

\subsection{Side-by-Side: Standard vs. Non-Standard Measurements}

\begin{center}
\begin{tabular}{|l|l|l|}
\hline
\textbf{Aspect} & \textbf{Standard Basis} & \textbf{Arbitrary Basis} \\
\hline
Basis vectors & $\ket{0}, \ket{1}$ & $\ket{\phi_0}, \ket{\phi_1}$ \\
\hline
Observable & Pauli $Z$ operator & Arbitrary Hermitian operator \\
\hline
Projectors & $\ket{0}\bra{0}, \ket{1}\bra{1}$ & $\ket{\phi_0}\bra{\phi_0}, \ket{\phi_1}\bra{\phi_1}$ \\
\hline
Implementation & Direct measurement & Apply $U^\dagger$, then measure \\
\hline
Hardware & Native operation & Requires gates \\
\hline
Information & Z-component & Arbitrary observable \\
\hline
\end{tabular}
\end{center}

\subsection{Key Conceptual Takeaways}

\begin{enumerate}
    \item \textbf{Hilbert Space:} Quantum playground where states live as vectors

    \item \textbf{Observables:} Hermitian operators representing measurable quantities

    \item \textbf{Born Rule:} $p(i) = \left|\braket{\phi_i}{\psi}\right|^2$ — the bridge between quantum amplitudes and classical probabilities

    \item \textbf{Basis-Change Equivalence:} Measurement in $\{\ket{\phi_i}\}$ = $U^\dagger$ + measurement in $\{\ket{i}\}$

    \item \textbf{Hardware Reality:} All quantum hardware measures only in computational basis

    \item \textbf{Hadamard Basis:} $\{\ket{+}, \ket{-}\}$ with $H$ as basis-change unitary

    \item \textbf{Partial Measurements:} Can measure subsystems in different bases
\end{enumerate}

\subsection{Applications in Upcoming Topics}

\begin{itemize}
    \item \textbf{Next lecture:} Deutsch-Jozsa algorithm uses $H$ before final measurement

    \item \textbf{Quantum algorithms:} Grover's algorithm, quantum Fourier transform

    \item \textbf{Quantum information:} Bell tests, quantum key distribution

    \item \textbf{Quantum error correction:} Syndrome measurements in different bases
\end{itemize}

\section{Exercises and Further Exploration}

\subsection{Problem Set: Basis Change Calculations}

\begin{enumerate}
    \item For state $\ket{\psi} = \frac{1}{\sqrt{3}}\ket{0} + \sqrt{\frac{2}{3}}\ket{1}$, calculate probabilities for measurement in:
    \begin{enumerate}
        \item Computational basis
        \item $\{\ket{+}, \ket{-}\}$ basis
        \item $\{\ket{+i}, \ket{-i}\}$ basis
    \end{enumerate}

    \item Construct the unitary $U$ that maps $\{\ket{0}, \ket{1}\}$ to basis $\{\frac{1}{2}\ket{0} + \frac{\sqrt{3}}{2}\ket{1}, \frac{\sqrt{3}}{2}\ket{0} - \frac{1}{2}\ket{1}\}$.

    \item Show that measuring $\ket{+}$ in $\{\ket{+}, \ket{-}\}$ basis always gives "+", while measuring $\ket{0}$ gives 50/50 outcomes.

    \item For the Bell state $\ket{\Phi^+}$, calculate the post-measurement state of qubit 2 when qubit 1 is measured in $\{\ket{+}, \ket{-}\}$ basis and outcome is "-".

    \item Prove that for any unitary $U$, measurement in basis $\{U\ket{0}, U\ket{1}\}$ is equivalent to applying $U^\dagger$ then measuring in computational basis.
\end{enumerate}

\subsection{Exploration: Other Important Bases}

\begin{itemize}
    \item \textbf{Y-basis:} $\{\ket{+i}, \ket{-i}\}$ with $U = HS^\dagger$

    \item \textbf{Breidbart basis:} $\{\cos(\pi/8)\ket{0} + \sin(\pi/8)\ket{1}, \sin(\pi/8)\ket{0} - \cos(\pi/8)\ket{1}\}$
    Used in optimal quantum cloning and cryptography.

    \item \textbf{Mutually unbiased bases (MUBs):} Bases where $|\braket{\phi_i}{\psi_j}|^2 = 1/d$ for all $i,j$.
    For qubits: X, Y, Z bases are mutually unbiased.
\end{itemize}

\subsection{Looking Ahead: POVMs and Generalized Measurements}

\begin{tcolorbox}[title=Advanced Preview]
Beyond projective measurements lie \textbf{POVMs (Positive Operator-Valued Measures)}:
\[
E_i \geq 0, \quad \sum_i E_i = I, \quad p(i) = \braopket{\psi}{E_i}{\psi}.
\]
POVMs can distinguish non-orthogonal states and are more general than projective measurements. We'll cover these in advanced topics.
\end{tcolorbox}

\begin{tcolorbox}[title=Final Thought: Why the Born Rule Matters]
The Born rule is not just a formula---it's the \textbf{interface between quantum and classical}. When we measure in different bases, we're asking different questions of nature. Each basis reveals different aspects of the quantum state, much like rotating a 3D object reveals different 2D shadows. This ability to "ask questions in different ways" is what gives quantum computing its power over classical computing.
\end{tcolorbox}

\end{document}
