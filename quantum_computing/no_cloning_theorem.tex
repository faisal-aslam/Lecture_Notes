\documentclass[12pt]{article}
\usepackage{amsmath, amssymb, amsthm}
\usepackage{physics}  % Already defines \ket, \bra, \braket
\usepackage{graphicx}
\usepackage{tcolorbox}
\usepackage{geometry}
\geometry{margin=1in}

% Remove the conflicting definitions since physics package provides them
% \newcommand{\ket}[1]{\left|#1\right\rangle}  % REMOVED - already defined
% \newcommand{\bra}[1]{\left\langle#1\right|}  % REMOVED - already defined
% \newcommand{\braket}[2]{\left\langle#1|#2\right\rangle}  % REMOVED - already defined

% Only define commands that physics package doesn't provide
\newcommand{\proj}[1]{\ket{#1}\bra{#1}}
\newcommand{\tensor}{\otimes}

\title{The No-Cloning Theorem in Quantum Computing}
\author{Quantum Computing Course}
\date{}

\begin{document}

\maketitle

\section{Introduction}

In classical computing, information can be copied freely. If you have a classical bit with value 0 or 1, you may duplicate it perfectly and arbitrarily many times. This ability underlies classical error correction, memory backup, and communication protocols.

Quantum information behaves fundamentally differently. In quantum mechanics, it is \emph{impossible} to perfectly copy an arbitrary unknown quantum state. This limitation is formalized in the \textbf{No-Cloning Theorem}, independently proved by Wootters and Zurek, and by Dieks, in 1982.

\subsection{What would be possible if we could clone qubits?}

If perfect cloning of unknown quantum states were possible, it would lead to dramatic (and unphysical) consequences:

\begin{enumerate}
    \item \textbf{Perfect quantum state estimation:} One could make many identical copies of an unknown state $\ket{\psi}$ and perform tomography to reconstruct it exactly.
    
    \item \textbf{Faster-than-light communication:} Combined with entanglement, cloning could be used to transmit information instantaneously, violating relativistic causality.
    
    \item \textbf{Trivial quantum error correction:} Errors could be corrected simply by restoring from a cloned backup.
    
    \item \textbf{Breaking quantum cryptography:} Security of protocols such as BB84 relies crucially on the impossibility of copying quantum states without disturbance.
    
    \item \textbf{Unlimited measurement precision:} Repeated measurements on cloned copies would eliminate fundamental quantum uncertainty.
\end{enumerate}

The impossibility of these consequences reveals a deep structural difference between classical and quantum information.

\section{Statement of the Theorem}

\begin{tcolorbox}[title=No-Cloning Theorem]
There does not exist a unitary operator $U$ and a fixed quantum state $\ket{\Sigma}$ (the \emph{blank} state) such that for \emph{all} normalized states $\ket{\psi}$,
\[
U(\ket{\psi} \tensor \ket{\Sigma}) = \ket{\psi} \tensor \ket{\psi}.
\]
\end{tcolorbox}

\textbf{Important clarification:}  
The theorem concerns \emph{unknown or arbitrary} quantum states. If the classical description of $\ket{\psi}$ is known, one may always prepare as many copies as desired. The impossibility lies in constructing a \emph{single physical device} that clones \emph{every} possible input state.

\section{First Proof: Linearity Argument}

\subsection{Setup}

Assume, for contradiction, that there exists a unitary operator $U$ and a fixed blank state $\ket{\Sigma}$ such that
\begin{equation}
U(\ket{\psi} \tensor \ket{\Sigma}) = \ket{\psi} \tensor \ket{\psi}
\label{eq:clone}
\end{equation}
for all quantum states $\ket{\psi}$.

\subsection{Proof}

Let $\ket{\alpha}$ and $\ket{\beta}$ be two distinct normalized quantum states. By assumption,
\begin{align}
U(\ket{\alpha} \tensor \ket{\Sigma}) &= \ket{\alpha} \tensor \ket{\alpha}, \label{eq:clonealpha}\\
U(\ket{\beta} \tensor \ket{\Sigma}) &= \ket{\beta} \tensor \ket{\beta}. \label{eq:clonebeta}
\end{align}

Consider their superposition
\[
\ket{\gamma} = a\ket{\alpha} + b\ket{\beta},
\]
with complex coefficients satisfying $\abs{a}^2 + \abs{b}^2 = 1$.

If cloning is universal, it must also clone $\ket{\gamma}$:
\begin{equation}
U(\ket{\gamma} \tensor \ket{\Sigma}) = \ket{\gamma} \tensor \ket{\gamma}.
\label{eq:clonepsi}
\end{equation}

By linearity of unitary operators,
\begin{align}
U(\ket{\gamma} \tensor \ket{\Sigma})
&= U\big[(a\ket{\alpha} + b\ket{\beta}) \tensor \ket{\Sigma}\big] \\
&= aU(\ket{\alpha} \tensor \ket{\Sigma}) + bU(\ket{\beta} \tensor \ket{\Sigma}) \\
&= a(\ket{\alpha} \tensor \ket{\alpha}) + b(\ket{\beta} \tensor \ket{\beta}).
\label{eq:linearresult}
\end{align}

On the other hand,
\begin{align}
\ket{\gamma} \tensor \ket{\gamma}
&= (a\ket{\alpha} + b\ket{\beta}) \tensor (a\ket{\alpha} + b\ket{\beta}) \\
&= a^2 \ket{\alpha\alpha}
+ ab \ket{\alpha\beta}
+ ba \ket{\beta\alpha}
+ b^2 \ket{\beta\beta}.
\label{eq:gammatensor}
\end{align}

Comparing \eqref{eq:linearresult} and \eqref{eq:gammatensor}, we see that equality requires:
\[
a^2 = a, \quad b^2 = b, \quad \text{and} \quad ab = 0.
\]

The condition $ab = 0$ implies either $a = 0$ or $b = 0$.
\begin{itemize}
    \item If $a = 0$, then from $b^2 = b$ and normalization $\abs{b}^2 = 1$, we get $b = 1$ (up to a phase), so $\ket{\gamma} = \ket{\beta}$.
    \item If $b = 0$, then from $a^2 = a$ and normalization $\abs{a}^2 = 1$, we get $a = 1$ (up to a phase), so $\ket{\gamma} = \ket{\alpha}$.
\end{itemize}

Thus, the only cases where cloning works are when $\ket{\gamma}$ is either $\ket{\alpha}$ or $\ket{\beta}$. The cloning machine fails for genuine superpositions.

\subsection{Conclusion of First Proof}

The linearity of quantum mechanics forbids a universal cloning transformation. A machine that clones even two distinct states cannot clone their superpositions.

\section{Second Proof: Inner Product Preservation}

\subsection{Setup}

Assume again the existence of a unitary $U$ such that
\[
U(\ket{\psi} \tensor \ket{\Sigma}) = \ket{\psi} \tensor \ket{\psi}
\]
for all $\ket{\psi}$.

\subsection{Proof}

Let $\ket{\phi}$ and $\ket{\psi}$ be two arbitrary states. Before applying $U$, the inner product is
\begin{equation}
\braket{\phi}{\psi} \braket{\Sigma}{\Sigma} = \braket{\phi}{\psi},
\label{eq:innerbefore}
\end{equation}
since $\ket{\Sigma}$ is normalized ($\braket{\Sigma}{\Sigma} = 1$).

After applying $U$, using unitarity ($U^\dagger U = I$) and the cloning assumption:
\begin{align}
&\left(\bra{\phi} \tensor \bra{\Sigma}\right) U^\dagger U \left(\ket{\psi} \tensor \ket{\Sigma}\right) \notag \\
&= \left(\bra{\phi} \tensor \bra{\phi}\right) \left(\ket{\psi} \tensor \ket{\psi}\right) \notag \\
&= \braket{\phi}{\psi} \cdot \braket{\phi}{\psi} \notag \\
&= \abs{\braket{\phi}{\psi}}^2.
\label{eq:innerafter}
\end{align}

Since unitary transformations preserve inner products, \eqref{eq:innerbefore} and \eqref{eq:innerafter} must be equal:
\[
\braket{\phi}{\psi} = \abs{\braket{\phi}{\psi}}^2.
\]

Let $x = \braket{\phi}{\psi}$. Writing $x = re^{i\theta}$ in polar form:
\[
re^{i\theta} = r^2.
\]

This implies:
\begin{enumerate}
    \item $r = r^2$ $\Rightarrow$ $r(1 - r) = 0$ $\Rightarrow$ $r = 0$ or $r = 1$
    \item If $r = 1$, then $e^{i\theta} = 1$ $\Rightarrow$ $\theta = 0$ (mod $2\pi$)
\end{enumerate}

Thus, $\braket{\phi}{\psi} = \abs{\braket{\phi}{\psi}}^2$ holds only if:
\begin{itemize}
    \item $\braket{\phi}{\psi} = 0$ (the states are orthogonal), or
    \item $\braket{\phi}{\psi} = 1$ (the states are identical, up to a global phase)
\end{itemize}

\subsection{Interpretation}

Perfect cloning is possible only for mutually orthogonal states or identical states. Since arbitrary quantum states are generally non-orthogonal, a universal cloner cannot exist.

\section{Consequences and Implications}

\subsection{What we \textit{can} do}

\begin{enumerate}
    \item \textbf{Cloning known orthogonal states:}  
    If a quantum state is guaranteed to belong to a known orthogonal set, it can be cloned.
    
    \item \textbf{Approximate cloning:}  
    Universal approximate cloners exist. The optimal qubit cloner (Bužek–Hillery, 1996) achieves fidelity $5/6$.
    
    \item \textbf{State-dependent cloning:}  
    One may design cloners optimized for specific ensembles of states.
\end{enumerate}

\subsection{Important applications}

\begin{enumerate}
    \item \textbf{Quantum key distribution (BB84):} The security relies on the fact that an eavesdropper cannot copy qubits without disturbing them.
    
    \item \textbf{Quantum money (Wiesner, 1983):} Uses the no-cloning theorem to prevent counterfeiting of quantum banknotes.
    
    \item \textbf{Quantum error correction:} Unlike classical error correction which often uses repetition, quantum error correction requires more sophisticated codes like the Shor code or surface codes.
    
    \item \textbf{Foundations of quantum information theory:} Highlights fundamental differences between classical and quantum information.
\end{enumerate}

\section{Examples and Counterexamples}

\subsection{Example 1: Cloning orthogonal states}

The CNOT gate clones computational basis states:
\[
U(\ket{0}\tensor\ket{0}) = \ket{0}\tensor\ket{0}, \quad
U(\ket{1}\tensor\ket{0}) = \ket{1}\tensor\ket{1}.
\]

But for $\ket{+} = (\ket{0} + \ket{1})/\sqrt{2}$,
\[
U(\ket{+}\tensor\ket{0}) = \frac{1}{\sqrt{2}}(\ket{00} + \ket{11}) \neq \ket{+}\tensor\ket{+}.
\]

\subsection{Example 2: Measurement and cloning}

Measurement converts quantum information into classical information, which can be copied freely — but at the cost of destroying the original quantum state. This illustrates the trade-off: you can "clone" by measuring, but only by destroying the original quantum coherence.

\section{Mathematical Perspective}

\subsection{Algebraic perspective}

Cloning would require a nonlinear map
\[
\ket{\psi} \mapsto \ket{\psi} \tensor \ket{\psi},
\]
which is incompatible with the linear structure of quantum mechanics. Unitary evolution (and more generally, completely positive maps) must be linear.

\subsection{Geometric perspective}

On the Bloch sphere, cloning would require mapping a single point (representing $\ket{\psi}$) to two identical points. Unitary transformations correspond to rotations of the entire Bloch sphere, which cannot map one point to two points.

\section{Historical Context}

The theorem was discovered relatively late in the development of quantum mechanics:
\begin{itemize}
    \item \textbf{Wootters and Zurek (1982):} "A single quantum cannot be cloned"
    \item \textbf{Dieks (1982):} "Communication by EPR devices"
\end{itemize}

The independent discoveries in the same year highlight how this fundamental limitation was not immediately obvious despite quantum mechanics being over 50 years old at the time.

\section{Summary}

\begin{tcolorbox}[title=Key Takeaways]
\begin{itemize}
    \item Arbitrary unknown quantum states cannot be perfectly cloned.
    
    \item The theorem follows from two fundamental properties of quantum mechanics:
    \begin{enumerate}
        \item Linearity of quantum evolution
        \item Preservation of inner products under unitary transformations
    \end{enumerate}
    
    \item Two equivalent proofs:
    \begin{enumerate}
        \item \textbf{Linearity proof:} Shows that cloning a superposition leads to contradictions
        \item \textbf{Inner product proof:} Uses the fact that unitaries preserve inner products
    \end{enumerate}
    
    \item Orthogonal states can be cloned if the orthogonal set is known in advance.
    
    \item Approximate cloning is possible (optimal fidelity $5/6$ for qubits), but perfect cloning is not.
    
    \item The theorem has profound implications for quantum cryptography, error correction, and our understanding of quantum information.
\end{itemize}
\end{tcolorbox}

\section*{Exercises}

\begin{enumerate}
    \item \textbf{No-deleting theorem:} Show that the no-cloning theorem implies a dual "no-deleting" theorem: there is no unitary operation that can delete an arbitrary unknown quantum state while preserving unitarity.
    
    \item \textbf{Optimal cloning fidelity:} For the Bužek–Hillery cloning machine, the optimal cloning fidelity for an arbitrary single qubit state is $F = \frac{5}{6}$. Verify that this satisfies $F > \frac{2}{3}$, which is the fidelity achieved by the simple "measure and prepare" strategy.
    
    \item \textbf{Cloning orthogonal sets:} Prove that if $\{\ket{\psi_i}\}$ is a set of mutually orthogonal quantum states, then there exists a unitary operator that clones all states in the set. Construct such a unitary explicitly for the set $\{\ket{0}, \ket{1}\}$.
    
    \item \textbf{Anti-linear "cloning":} Consider the transformation $\ket{\psi} \mapsto \ket{\psi} \tensor \ket{\psi^*}$, where $\ket{\psi^*}$ is the complex conjugate of $\ket{\psi}$. Show that this map is anti-linear and therefore not physically implementable as a unitary operation.
    
    \item \textbf{No-broadcasting theorem:} Research and explain how the no-cloning theorem generalizes to mixed states as the "no-broadcasting" theorem. What is the key difference between cloning pure states and broadcasting mixed states?
\end{enumerate}

\end{document}
