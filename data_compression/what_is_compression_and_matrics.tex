\section{Introduction to Data Compression}

\subsection{Learning Objectives}
By the end of this lecture, students will be able to:
\begin{itemize}
    \item Understand the motivation and benefits of data compression
    \item Differentiate between lossless and lossy compression techniques
    \item Compute and interpret common compression performance metrics
    \item Apply the Huffman coding algorithm step by step
    \item Analyze real-world compression trade-offs
\end{itemize}

\subsection{Introduction and Motivation: Why Compress Data?}

Data compression is the process of representing information using fewer bits than its original form.
It is a fundamental component of modern computing systems, enabling efficient storage, faster communication,
and reduced operational costs.

Everyday applications of compression include:
\begin{itemize}
    \item Streaming audio and video
    \item Image storage and sharing
    \item File archiving and backups
    \item Network communication and cloud services
\end{itemize}

\begin{definitionbox}
\textbf{Data Compression} is the process of reducing the number of bits required to represent information, either:
\begin{itemize}
    \item \textbf{Losslessly}: allowing exact reconstruction of the original data
    \item \textbf{Lossily}: allowing controlled loss of information to achieve higher compression
\end{itemize}
\end{definitionbox}

\subsubsection{Benefits of Data Compression}
Data compression provides three key benefits that are critical in modern computing:

\begin{enumerate}
    \item \textbf{Reduce Storage Space}:
    \begin{itemize}
        \item Allows more data to be stored in the same physical space
        \item Enables archival of historical data that would otherwise be discarded
        \item Reduces hardware requirements for storage systems
    \end{itemize}

    \item \textbf{Reduce Communication Time and Bandwidth}:
    \begin{itemize}
        \item Enables faster file transfers and downloads
        \item Makes high-quality streaming (4K/8K video) practical over limited bandwidth
        \item Reduces latency in real-time applications like video conferencing and online gaming
        \item Allows IoT devices to transmit data efficiently over wireless networks
    \end{itemize}

    \item \textbf{Save Money}:
    \begin{itemize}
        \item Reduces cloud hosting costs (storage and egress fees)
        \item Lowers communication costs for data transmission
        \item Decreases capital expenditure on storage hardware
        \item Reduces energy consumption for data centers and network infrastructure
    \end{itemize}

\end{enumerate}


\subsection{Lossless vs.\ Lossy Compression}

\subsubsection{Lossless Compression}
Lossless compression guarantees perfect reconstruction of the original data.
It is essential when accuracy and data integrity are critical.

\textbf{Typical applications:}
\begin{itemize}
    \item Text files and source code
    \item Executables and databases
    \item Medical, scientific, and legal data
\end{itemize}

\subsubsection{Lossy Compression}
Lossy compression achieves higher compression ratios by discarding information that is less perceptible or less important.

\textbf{Typical applications:}
\begin{itemize}
    \item Audio (MP3, AAC)
    \item Images (JPEG)
    \item Video (H.264, H.265)
\end{itemize}

\subsubsection{Choosing Between Lossless and Lossy}

\begin{table}[htbp]
\centering
\begin{tabularx}{\textwidth}{|p{3.5cm}|X|X|}
\hline
\textbf{Factor} & \textbf{Lossless Compression} & \textbf{Lossy Compression} \\
\hline
Reconstruction & Exact & Approximate \\
\hline
Data sensitivity & High & Moderate to low \\
\hline
Typical ratios & Low to moderate & High \\
\hline
Quality impact & None & Controlled degradation \\
\hline
\end{tabularx}
\caption{Lossless vs.\ Lossy Compression}
\end{table}

\subsection{Compression Performance Metrics}

\subsubsection{Size-Based Metrics}

\begin{align*}
\text{Compression Ratio (CR)} &= \frac{\text{Original Size}}{\text{Compressed Size}} \\
\text{Compression Factor} &= \frac{\text{Compressed Size}}{\text{Original Size}} \\
\text{Space Savings (\%)} &= \left(1 - \frac{\text{Compressed Size}}{\text{Original Size}}\right) \times 100\%
\end{align*}

\textbf{Interpretation:}
\begin{itemize}
    \item Larger compression ratios indicate better compression
    \item Smaller compression factors indicate better compression
\end{itemize}

\subsubsection{Rate-Based Metrics}

\begin{align*}
\text{Bits per Sample (bps)} &= \frac{\text{Compressed Size (bits)}}{\text{Number of samples}} \\
\text{Bit-rate (bps)} &= \frac{\text{Compressed Size (bits)}}{\text{Time (seconds)}}
\end{align*}

These metrics are particularly important in audio and video compression systems.

\subsection{Worked Example: Audio Compression Metrics}

\begin{examplebox}
\textbf{Uncompressed Audio Properties}
\begin{itemize}
    \item Duration: 180 seconds
    \item Sampling rate: 44.1 kHz
    \item Bit depth: 16 bits
    \item Channels: 2 (stereo)
\end{itemize}

\textbf{Original Size Calculation}
\[
\begin{aligned}
\text{Total samples} &= 180 \times 44{,}100 \times 2 = 15{,}876{,}000 \\
\text{Size (bits)} &= 15{,}876{,}000 \times 16 = 254{,}016{,}000 \\
\text{Size (MB)} &= \frac{254{,}016{,}000}{8 \times 1{,}048{,}576} \approx 30.27
\end{aligned}
\]

\textbf{Compression Results}
\begin{center}
\begin{tabular}{|l|c|c|c|c|}
\hline
\textbf{Method} & \textbf{Size (MB)} & \textbf{CR} & \textbf{Savings} & \textbf{Bit-rate} \\
\hline
FLAC (lossless) & 18.16 & 1.67:1 & 40\% & 807 kbps \\
MP3 @ 320 kbps & 6.75 & 4.49:1 & 77.7\% & 320 kbps \\
AAC @ 256 kbps & 5.40 & 5.61:1 & 82.2\% & 256 kbps \\
\hline
\end{tabular}
\end{center}
\end{examplebox}

\subsection{Huffman Coding}

Huffman coding is a widely used \textbf{lossless compression algorithm} that assigns variable-length binary codes to symbols based on their frequencies.
More frequent symbols receive shorter codes.

\subsubsection{Step-by-Step Huffman Coding Example}

\begin{examplebox}
\textbf{Message:} \texttt{MISSISSIPPI RIVER} (17 characters including space)

\textbf{Symbol Frequencies}

\begin{center}
\begin{tabular}{|c|c|}
\hline
\textbf{Symbol} & \textbf{Frequency} \\
\hline
I & 5 \\
S & 4 \\
P & 2 \\
R & 2 \\
M & 1 \\
V & 1 \\
E & 1 \\
(space) & 1 \\
\hline
\end{tabular}
\end{center}

\textbf{Tree Construction}
\begin{enumerate}
    \item Combine M(1) + V(1) $\rightarrow$ 2
    \item Combine E(1) + (space)(1) $\rightarrow$ 2
    \item Combine P(2) + R(2) $\rightarrow$ 4
    \item Combine 2 + 2 $\rightarrow$ 4
    \item Combine 4 + 4 $\rightarrow$ 8
    \item Combine I(5) + S(4) $\rightarrow$ 9
    \item Combine 8 + 9 $\rightarrow$ 17
\end{enumerate}

\textbf{One Possible Code Assignment}

\begin{center}
\begin{tabular}{|c|c|c|}
\hline
\textbf{Symbol} & \textbf{Code} & \textbf{Length} \\
\hline
I & 00 & 2 \\
S & 01 & 2 \\
P & 100 & 3 \\
R & 101 & 3 \\
M & 1100 & 4 \\
V & 1101 & 4 \\
E & 1110 & 4 \\
(space) & 1111 & 4 \\
\hline
\end{tabular}
\end{center}

\textbf{Compressed Size}
\[
5(2) + 4(2) + 2(3) + 2(3) + 4(1) = 52 \text{ bits}
\]

\textbf{Original Size (ASCII)} = $17 \times 8 = 136$ bits

\textbf{Compression Ratio} = $136 / 52 \approx 2.62:1$
\end{examplebox}

\subsubsection{Key Properties of Huffman Coding}
\begin{itemize}
    \item Produces prefix-free codes
    \item Enables instantaneous decoding
    \item Guarantees minimum average code length among prefix codes
    \item Widely used in practical compression systems
\end{itemize}


\subsection{End of Chapter Questions}

\begin{exercisebox}
\textbf{Problem 1: Basic Compression Metrics}

An uncompressed grayscale image has the following properties:
\begin{itemize}
    \item Resolution: $1024 \times 1024$ pixels
    \item Bit depth: 8 bits per pixel
\end{itemize}

After compression, the image size is 320 KB.

Calculate:
\begin{enumerate}[label=(\alph*)]
    \item Original image size in KB
    \item Compression ratio
    \item Compression factor
    \item Space savings percentage
\end{enumerate}
\end{exercisebox}

\begin{exercisebox}
\textbf{Problem 2: Audio Bit-rate and Storage}

A mono audio recording has the following parameters:
\begin{itemize}
    \item Duration: 5 minutes
    \item Sampling rate: 48 kHz
    \item Bit depth: 16 bits
\end{itemize}

The file is compressed using a lossy codec to a constant bit-rate of 192 kbps.

Calculate:
\begin{enumerate}[label=(\alph*)]
    \item Size of the uncompressed audio file in MB
    \item Size of the compressed file in MB
    \item Compression ratio
    \item Bits per sample after compression
\end{enumerate}
\end{exercisebox}

\begin{exercisebox}
\textbf{Problem 3: Comparing Compression Options}

A video clip has an uncompressed data rate of 120 Mbps.
Three compression options are available:

\begin{center}
\begin{tabular}{|c|c|}
\hline
\textbf{Option} & \textbf{Compressed Bit-rate} \\
\hline
A & 6 Mbps \\
B & 3 Mbps \\
C & 1.5 Mbps \\
\hline
\end{tabular}
\end{center}

For each option, calculate:
\begin{enumerate}[label=(\alph*)]
    \item Compression ratio
    \item Data consumed for a 10-minute video (in MB)
\end{enumerate}

Which option would you choose for:
\begin{enumerate}[label=(\roman*)]
    \item Live video streaming?
    \item Archival storage?
\end{enumerate}
Briefly justify your answers.
\end{exercisebox}

\begin{exercisebox}
\textbf{Problem 4: Huffman Coding Construction}

Given the following symbol frequencies:

\begin{center}
\begin{tabular}{|c|c|}
\hline
\textbf{Symbol} & \textbf{Frequency} \\
\hline
A & 10 \\
B & 8 \\
C & 6 \\
D & 5 \\
E & 4 \\
F & 3 \\
G & 2 \\
H & 2 \\
\hline
\end{tabular}
\end{center}

\begin{enumerate}[label=(\alph*)]
    \item Construct the Huffman tree step by step
    \item Assign a binary code to each symbol
    \item Compute the total number of bits required to encode the message
    \item Calculate the average number of bits per symbol
\end{enumerate}
\end{exercisebox}

\begin{exercisebox}
\textbf{Problem 5: Fixed-Length vs.\ Huffman Coding}

Using the symbol set from Problem~4:

\begin{enumerate}[label=(\alph*)]
    \item Determine the minimum fixed-length code required
    \item Compute the total number of bits using fixed-length coding
    \item Compare the result with Huffman coding
    \item Calculate the percentage reduction in total bits achieved by Huffman coding
\end{enumerate}
\end{exercisebox}

\begin{exercisebox}
\textbf{Problem 6: Text Compression Scenario}

A text file contains 50,000 characters and is stored using 8-bit ASCII encoding.

After compression using a lossless algorithm, the file size becomes 18 KB.

Calculate:
\begin{enumerate}[label=(\alph*)]
    \item Original file size in KB
    \item Compression ratio
    \item Compression factor
    \item Space savings percentage
\end{enumerate}

Explain why compression ratios for text files vary significantly depending on content.
\end{exercisebox}

\begin{exercisebox}
\textbf{Problem 7: Practical Design Question}

You are designing a compression system for a wearable health-monitoring device that:
\begin{itemize}
    \item Records sensor data continuously
    \item Has limited storage capacity
    \item Requires exact data reconstruction
    \item Operates on a low-power processor
\end{itemize}

\begin{enumerate}[label=(\alph*)]
    \item Should the system use lossless or lossy compression? Explain.
    \item Which performance metrics are most important in this scenario?
    \item Would a variable-length coding scheme be appropriate? Why or why not?
\end{enumerate}
\end{exercisebox}

